\section{K-Map Incrementing Logic}
The incrementing logic for each display is implemented using decade counters. For the unit's place of the seconds, the logic is as follows:
\section{Control Implementation}
\begin{enumerate}
    \item Pressing the first button will pause/play the clock.
    \item Pressing the second button while paused will increment the seconds.
    \item Pressing the third button while paused will increment the minutes.
    \item Pressing the fourth button while paused will increment the hours.
\end{enumerate}

To increment the minutes, the incrementing logic is run 60 times. Similarly, incrementing the hours requires running the loop 3600 times.


\begin{align}
    A_1 &= \overline{W_1}; \\
    B_1 &= (W_1 \land \overline{X_1} \land \overline{Z_1}) \lor (\overline{W_1} \land X_1);\\
    C_1 &= (\overline{X_1} \land Y_1) \lor (\overline{W_1} \land Y_1) \lor (W_1 \land X_1 \land \overline{Y_1});\\
    D_1 &= (\overline{W_1} \land Z_1) \lor (W_1 \land X_1 \land Y_1).
\end{align}
For the ten's place of the seconds, which varies from 0 to 5:
\section{Control Implementation}
\begin{enumerate}
    \item Pressing the first button will pause/play the clock.
    \item Pressing the second button while paused will increment the seconds.
    \item Pressing the third button while paused will increment the minutes.
    \item Pressing the fourth button while paused will increment the hours.
\end{enumerate}

To increment the minutes, the incrementing logic is run 60 times. Similarly, incrementing the hours requires running the loop 3600 times.


\begin{align}
    A_2 &= \overline{W_2};\\
    B_2 &= (\overline{Y_2} \land \overline{X_2} \land W_2) \lor (\overline{W_2} \land X_2);\\
    C_2 &= (\overline{W_2} \land Y_2) \lor (X_2 \land W_2);\\
    D_2 &= 0.
\end{align}

To synchronize the ten's place increment with the unit's place reaching 9, an additional variable $C$ is used:
\begin{align}
    C &= W_1 \land \overline{X_1} \land \overline{Y_1} \land Z_1 \\
    A_2 &= (A_2 \land C) \lor (W_2 \land \overline{C}) \\
    B_2 &= (B_2 \land C) \lor (X_2 \land \overline{C}) \\
    C_2 &= (C_2 \land C) \lor (Y_2 \land \overline{C}) \\
    D_2 &= (D_2 \land C) \lor (Z_2 \land \overline{C}).
\end{align}

Now, using the above logic, the ten's digit of seconds only updates when the unit's digit previously was 9. This logic can be reapplied for the next display, i.e., the unit's digit of the minutes:
\begin{align}
    A_3 &= \overline{W_3}; \\
    B_3 &= (W_3 \land \overline{X_3} \land \overline{Z_3}) \lor (\overline{W_3} \land X_3);\\
    C_3 &= (\overline{X_3} \land Y_3) \lor (\overline{W_3} \land Y_3) \lor (W_3 \land X_3 \land \overline{Y_3});\\
    D_3 &= (\overline{W_3} \land Z_3) \lor (W_3 \land X_3 \land Y_3).
\end{align}

The value of $C$ for this case is:
\begin{align}
    C &= W_2 \land \overline{X_2} \land Y_2 \land \overline{Z_2} \land W_1 \land \overline{X_1} \land \overline{Y_1} \land Z_1.
\end{align}
